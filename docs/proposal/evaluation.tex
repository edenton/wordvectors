\section{Evaluation}\label{sec:evaluation}

Goal 2 requires a concrete method of determining whether or not any of the subspaces correspond to meaningful concepts. 
We can use an existing word hierarchy, such as the hierarchy used in Imagenet \footnote{\url{http://www.image-net.org/}} or more generally  WordNet \footnote{\url{http://wordnet.princeton.edu/wordnet/}}, to test this. 
We propose evaluating a given subspace by considering where each of the words assigned to the subspace lie within the hierarchy.
If the majority of the words fall under a single superclass then we can conclude that the subspace does indeed correspond to a loosely related concept. 
We can then compute a score for each subspace based on the depth of the deepest node that contains a significant portion of the words in the subspace, if such a node exists. This metric makes sense since deeper nodes denotes a more specific concept in the hierarchy. 

We can test goal 3(a) this by generating a set of ambiguous words and computing nearest neighbors of the word vector after projecting it onto multiple relevant subspaces. 
We will evaluate goal 3(b) on a publicly available analogical reasoning dataset \cite{mikolov3}. 
The dataset consists of both semantic and syntactic questions of the form 'A is to B as C is to D'.
The task is to find D given A, B and C. 
As described in section \ref{sec:overview}, this task can be solved with a simple vector algebra in the embedding space.
We will test to see if performance can be improved by projecting vectors onto subspaces that are good approximations. 
