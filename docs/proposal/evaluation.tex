We propose two ways to evaluate the performance of our goals. 
In [3], the authors release a test set to measure relational similarity. 
Given a relationship of the form "A is to B as C is to D", the task involves finding D given A,B and C. 
Mikolov et al. 
solve this by doing simple vector algebra; they compute the vector representation of A-B+C and search for the nearest neighbor of the resulting vector. 
Their model outperforms baseline models at this task. 
We intend to test our performance on this “analogical reasoning” test. 
Our hypothesis is that the performance on analogical questions would be improved if points that can be well approximated by a lower dimensional hyperplane are projected onto the plane before doing the vector algebra. 
The second evaluation metric that considers how well the subspaces we find correspond to superclasses in an existing word hierarchy[8]. 
If the distribution of words in a subspace to superclasses in a hierarchy is largely unimodal, we can conclude that the subspaces do indeed correspond to loosely related concepts. 
